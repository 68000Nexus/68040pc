\documentclass{article}

%\usepackage[letterpaper, margin=1in]{geometry}
\usepackage[letterpaper]{geometry}
\usepackage{bytefield}
\usepackage{tabularx}
\usepackage{threeparttable}
\usepackage{xifthen}
\usepackage{xparse}
\usepackage{xspace}

\newcommand{\systemname}[0]{Baby040\xspace}

\makeatletter
\newcommand*{\textoverline}[1]{$\overline{\hbox{#1}}\m@th$}
\makeatother

\NewDocumentCommand{\addr}{mm}{%
  \IfNoValueTF{#2}
  	{\texttt{\$#1}}
	{\mbox{\texttt{\$#1}-\texttt{\$#2}}}
}

\newcommand{\memsection}[4]{
	\bytefieldsetup{bitheight=#3\baselineskip}
	\bitbox[]{10}{
		\texttt{#1}
		\\ \vspace{#3\baselineskip} \vspace{-2\baselineskip} \vspace{- #3pt}
		\texttt{#2}
	}
	\bitbox{16}{#4}
}

\setlength\parindent{0pt}
\begin{document}

\begin{titlepage}
	\begin{center}
		\vspace*{1cm}

		\Huge
		\textbf{\systemname Specification}

		\vspace{0.5cm}
		\LARGE
		A Personal Computer Platform

		\vspace{1.5cm}
		\textbf{Anthony Guerrero}

		\vfill

		\normalsize
		Document Revision 0.0.1

		\vspace{0.8cm}
	\end{center}
\end{titlepage}

\section{System Overview}

\subsection{Design Goal}
The primary objective of the \systemname is to construct a personal computer
platform that aligns with the capabilities and aesthetics of computing systems
from the 1990s. The system is not merely a reflection of retro hardware but a
modern interpretation emphasizing complete transparancy, user understanding,
and modifiability. The design principles prioritize open-source documentation,
source code, and hardware description language (HDL) to empower users and
enthusiasts to understand, modify, and expand upon the system.

\subsection{Key Features}
\begin{itemize}
	\item \textbf{Open Architecture:} An open-source ethos is at the core of the
	project. All documentation, designs, and code associated with this
	system will be freely accessible and modifiable by the end-users and
	community.

	\item \textbf{MC68040 Processor:} Serving as the heart of the system, this 32-bit
	microprocessor offers a blend of performance and functionality suitable
	for a wide range of computing tasks. It's Instruction Set Architecture
	(ISA) and built-in MMU lends itself well to implementing UNIX-like
	operating systems.

	\item \textbf{Flexible Expansion:} The platform is designed with expansion and
	modularity in mind, featuring an ``Expansion Bus'' architecture to
	facilitate the addition of new components. The bus dynamically assigns
	memory space to bus members, bypassing address conflicts and setup
	jumpers.

	\item \textbf{Integrated FPGA:} Utilizing the iCE40-HX4K FPGA, with full access
	to the CPU bus, most glue logic and peripheral controllers can be
	implemented and modified after the design is finalized.

	\item \textbf{512MB Dynamic RAM:} With four 72-pin SIMM slots, the platform
	supports up to 512MB of main system memory. (3.3v only!)

	\item \textbf{Standard Form Factor:} The PCB will be sized to fit in a pre-ATX
	form factor called ``Baby AT''. This allows for eight expansion slots,
	a DIN-5 sized cutout for a keyboard, and simple power management.
\end{itemize}

\subsection{System Philosophy}

Beyond the technical specifications, this project embodies a philosophy of
transparency, education, and user empowerment. The system is not just a
computing platform but a testament to the ethos that technology should be
understandable, modifiable, and, above all, open. Whether for education,
nostalgia, or innovation, this platform seeks to be a canvas for users to
explore the intricacies of computing, reminiscent of an era when personal
computing was just blossoming.

\section{Hardware}

\subsection{CPU}

The Motorola MC68040 was chosen because I found it in a box. The processor
operates in the non-multiplexed address mode and with control circuitry that
asserts the . Data Latch Enable (DLE) is not used.

\subsection{SRAM}

There are 512KB of SRAM for boot ROM, et al...

\subsection{DRAM}

Four 72-pin SIMM slots provide the bulk RAM for the system. Each SIMM's
\textoverline{RAS} lines are tied together and sent to the memory controller,
\textoverline{CAS} lines are routed separately. The memory controller only
supports 3.3v SIMMs (TODO)

The SIMMs I have (Keystron MK16DS3232LP-50) have (16) 16M $\times$ 4-bit DRAMs
per module. (Place chip to \textoverline{CAS}, \textoverline{RAS}, DQ here)

\subsection{I2C Bus}

I2C is used on-board for communication with the Real-Time Clock (DS1307)

\section{Memory Map}

Pay no attention to the memory map diagram at Figure \ref{map:sysmem}, it's
probably inaccurate.

\vspace{12pt}

When \textoverline{RESET} is asserted from a reboot, power-up, etc., the address
decoder maps addresses \mbox{\texttt{\$00000000-\$000003FF}} to
\mbox{\texttt{\$8000FC00-\$8000FFFF}}. This allows the processor to read the
exception vector table from ROM, but writes to these addresses will be routed to
RAM. Once the system RAM test passes, the boot program copies the table to
system RAM and writes to a configuration register which remaps all accesses from
\texttt{\$00000000} to RAM.

\begin{figure}[]
	\caption{System Memory Map}
	\label{map:sysmem}
	\vspace{2ex}
	\begin{bytefield}{12}
		\memsection{0000 0000}{1fff ffff}{12}{DRAM}\\
		\memsection{2000 0000}{200f ffff}{3}{SRAM}\\
		\memsection{}{}{6}{-}\\
		\memsection{8000 0000}{8000 ffff}{2}{ROM}\\
		\memsection{}{ffff ffff}{3}{-}\\
	\end{bytefield}
\end{figure}

\section{On-board Bus}

Standard 68040 bus, DRAM data is buffered through FPGA, MC68150 bus resizer for
expansion bus, I2C provided by PCA9564...

\section{Expansion Bus}

The system provides a Zorro III-inspired expansion slot bus, the main feature of
which is dynamic memory mapping of add-in devices. 

\subsection{Signal Description}

\subsection{Electrical Specifications}

\begin{figure}[h]
	\begin{centering}

	\begin{threeparttable}
	\caption{Expansion Bus Connector Pinout}
	\begin{tabularx}{\textwidth}
		{| c | X || c | X || c | X |}
		\hline
		Pin & Name & Pin & Name & Pin & Name \\
		\hline\hline
		a1  & NC\tnote{1}	& b1  &	NC		& c1  &	NC 		\\
		\hline
		a2  & NC		& b2  &	NC		& c2  &	NC 		\\
		\hline
		a3  & NC		& b3  &	NC		& c3  &	NC 		\\
		\hline
		a4  & NC		& b4  &	NC		& c4  &	NC 		\\
		\hline
		a5  & NC		& b5  &	NC		& c5  &	NC 		\\
		\hline
		a6  & NC		& b6  &	NC		& c6  &	NC 		\\
		\hline
		a7  & NC		& b7  &	NC		& c7  &	NC 		\\
		\hline
		a8  & NC		& b8  &	NC		& c8  &	NC 		\\
		\hline
		a9  & NC		& b9  &	NC		& c9  &	NC 		\\
		\hline
		a10 & NC		& b10 &	NC		& c10 &	NC 		\\
		\hline
		a11 & NC		& b11 &	NC		& c11 &	NC 		\\
		\hline
		a12 & NC		& b12 &	NC		& c12 &	NC 		\\
		\hline
		a13 & NC		& b13 &	NC		& c13 &	NC 		\\
		\hline
		a14 & NC		& b14 &	NC		& c14 &	NC 		\\
		\hline
		a15 & NC		& b15 &	NC		& c15 &	NC 		\\
		\hline
		a16 & NC		& b16 &	NC		& c16 &	NC 		\\
		\hline
		a17 & NC		& b17 &	NC		& c17 &	NC 		\\
		\hline
		a18 & NC		& b18 &	NC		& c18 &	NC 		\\
		\hline
		a19 & NC		& b19 &	NC		& c19 &	NC 		\\
		\hline
		a20 & NC		& b20 &	NC		& c20 &	NC 		\\
		\hline
		a21 & NC		& b21 &	NC		& c21 &	NC 		\\
		\hline
		a22 & NC		& b22 &	NC		& c22 &	NC 		\\
		\hline
		a23 & NC		& b23 &	NC		& c23 &	NC 		\\
		\hline
		a24 & NC		& b24 &	NC		& c24 &	NC 		\\
		\hline
		a25 & NC		& b25 &	NC		& c25 &	NC 		\\
		\hline
		a26 & NC		& b26 &	NC		& c26 &	NC 		\\
		\hline
		a27 & NC		& b27 &	NC		& c27 &	NC 		\\
		\hline
		a28 & NC		& b28 &	NC		& c28 &	NC 		\\
		\hline
		a29 & NC		& b29 &	NC		& c29 &	NC 		\\
		\hline
		a30 & NC		& b30 &	NC		& c30 &	NC 		\\
		\hline
		a31 & NC		& b31 &	NC		& c31 &	NC 		\\
		\hline
		a32 & NC		& b32 &	NC		& c32 &	NC 		\\
		\hline
	\end{tabularx}
	\begin{tablenotes}
	\item [1] This means it's not connected.
	\end{tablenotes}
	\end{threeparttable}
	\end{centering}
\end{figure}
		
\subsection{Mechanical Specification}

\section{Interrupt System}

\section{Special Thanks}

\textbf{Ben Eater} for creating engaging videos on computer engineering,
inspiring me to build a computer of my own.

\textbf{Lawrence Manning} MAXI030

\textbf{Stephen Moody} Y Ddraig(030)

\textbf{Dave Haynie} Author of "The Zorro III Bus Specification"

\end{document}
